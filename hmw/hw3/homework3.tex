\documentclass[12pt]{article}
\usepackage[margin=1in]{geometry} 
\usepackage{amsmath,amsthm,amssymb,amsfonts}
\usepackage{hyperref,enumerate}
\hypersetup{
    colorlinks=true,
    linkcolor=blue,
    filecolor=magenta,      
    urlcolor=cyan,
}
\urlstyle{same} 

\newcommand{\N}{\mathbb{N}}
\newcommand{\Z}{\mathbb{Z}}
 
\newenvironment{problem}[2][Problem:]{\begin{trivlist}
\item[\hskip \labelsep {\bfseries #1}\hskip \labelsep {\bfseries #2.}]}{\end{trivlist}}
%If you want to title your bold things something different just make another thing exactly like this but replace "problem" with the name of the thing you want, like theorem or lemma or whatever
 
\begin{document}
 
%\renewcommand{\qedsymbol}{\filledbox}
%Good resources for looking up how to do stuff:
%Binary operators: http://www.access2science.com/latex/Binary.html
%General help: http://en.wikibooks.org/wiki/LaTeX/Mathematics
%Or just google stuff


\title{ENGS/QBS 108 Fall 2017 Assignment 3}
\author{Due October 17, 2017 \\ Instructors: George Cybenko and Saeed Hassanpour \\ Prepared by: Benjamin Priest}
\date{}
\maketitle

\pagebreak

%\section{$K$ means clustering and $K$ nearest neighbors classification}
\begin{problem}{Porto Seguro Dataset [75 points]}
In this problem, you will explore and attempt to solve a realistic machine learning problem hosted on the \href{https://www.kaggle.com/}{Kaggle} competition platform.
In order to download the data and access other competition resources, you will need to create a kaggle account. 
The \href{https://www.kaggle.com/c/porto-seguro-safe-driver-prediction/data}{Porto Seguro dataset} includes a number of anonymized features of vehicle drivers collected over the course of a year, as well as a target flag indicating whether the driver filled an insurance claim during the year.
Your task is to train models to predict driver safety using this data. 

\begin{enumerate}
    \item {[10 points]} 
    Data analysis is an important first step. 
    The first two fields in train.csv, ``id'' and ``target,'' are an index and the label, respectively. 
    The other 57 fields are features with partially anonymized names. 
    The elements of the ``target'' column are in $\{0,1\}$, where $1$ indicates the driver filed an insurance claim during the year.
    Proceed to explore this problem's dataset by addressing the following:
    \begin{enumerate}
            \item What are the dimensions of the training and testing datasets?
            \item Features ending in ``\_cat'' are categorical, meaning that their entries are categories rather than numerals. How many categories does each such feature include?
            \item Features ending in ``\_bin'' are binary, taking entries in $\{0,1\}$. Are any of these features ``one-hot'' encodings of a single categorical feature? If so, which ones, and how many categories does it include? 
            \item Explore the \href{https://www.kaggle.com/c/porto-seguro-safe-driver-prediction/discussion}{discussion} page of the competition. What else can you learn about the features or dataset from what others have done? Include code if relevant.
    \end{enumerate}

    \item {[20 points]} 
    In this part, you will apply an SVM to solve a classification problem using the Porto Seguro dataset.
    You will design an SVM that reads a feature vector and predicts whether that driver will file an insurance claim.
    \begin{enumerate}
    	\item Hold out 30\% of the training dataset as the test set for this problem. Separate the remaining training data into a \href{https://en.wikipedia.org/wiki/Cross-validation_(statistics)#k-fold_cross-validation}{$k$-fold cross validation} testing framework with $k$=4. \label{2a}
    	\item Train SVM models using your validation framework. Try a few different kernels. What kernel yields the best classification accuracy? Note: Training will most likely take a \emph{long} time. 
	\item Apply your best model to the testing set you partitioned out in \ref{2a}. Report your classification accuracy. 
    \end{enumerate}

    \item {[20 points]} \label{3}
    In this part, you will attempt proscribed a proscribed solution to the competition. 
    You will build a (simple) deep neural net whose desired output is a probability that a driver will file a claim given the input feature vector. 
    You will report results in terms of a \href{https://en.wikipedia.org/wiki/Receiver_operating_characteristic}{ROC curve}, where you vary the decision threshold between 0 and 1. 
    \begin{enumerate}
    	\item Hold out 20\% of the training dataset as the validation set for this problem.
    	\item Construct a deep neural network with three dense layers of 100, 50, and 20 neurons, respectively, as well as a single sigmoidally activated output layer.
	\item Train your model on the remaining training data using \href{https://en.wikipedia.org/wiki/Cross_entropy}{cross-entropy loss}. Report your ROC curve on the training data.
	\item Apply your model to the validation set and report the corresponding ROC curve. 
	\item Retrain your model on the \emph{whole} training set and apply your model to the testing set. The output format is specified in sample\_submission.csv. \href{https://www.kaggle.com/c/porto-seguro-safe-driver-prediction/submit}{Submit} your output to Kaggle. What performance does it return?
    \end{enumerate}

    \item {[20 points]} \label{4}
    In this part, you will attempt your own solution to the competition. 
    Repeat problem \ref{3} with a model of your choice.
    You are free to browse \href{https://www.kaggle.com/c/porto-seguro-safe-driver-prediction/kernels}{existing kernels} for ideas, but your work should ultimately be your own. 
    Credit any resources that you use referenced extensively in your solution.
    
    \item {[5 points]}
    How do your solutions in problems \ref{3} and \ref{4} compare? Explain your observations. 



\end{enumerate}
\end{problem}




\end{document}